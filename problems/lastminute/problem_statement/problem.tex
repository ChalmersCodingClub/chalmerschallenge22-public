\problemname{Last Minute}

Ono! Chalmers Challenge is only one day away and
\href{https://codingclub.chs.chalmers.se/}{\textsc{Chalmers Coding Club}}
haven’t had the time to prepare all tasks. As we all know, competitive
programming problems are created by combining funny problem statements with
computer programming methods. There are $A_{uniq}+A_{reuse}$ problem statement
themes and $B_{uniq}+B_{reuse}$ algorithmic methods. Some of these are one-trick
ponies that only work once while others are so good they never get old. Out of
all the themes and methods, only $A_{reuse}$ and $B_{reuse}$ are good enough to
be reused continually. How many different problems can be created by combining
all these themes and methods?

\section*{Input}
Input consists of one line with four space-separated integers: $0 \le A_{uniq},
B_{uniq}, A_{reuse}, B_{reuse} \le 10^9$, the number of single-use themes and
methods and the number of reusable themes and methods respectively.

\section*{Output}
Output a single integer, the number of problems that can be created in total.
