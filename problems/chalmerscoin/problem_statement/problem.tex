\problemname{Chalmers Coin}
The industrial engineering and management division at Chalmers have recently
invented something they think will revolutionize money. They have invented a
brand new cryptocurrency called the Chalmers Coin (CC)! To mine a block of CC
you need to find a sufficiently small value of $\texttt{hash}(x \,.\, k)$ where
$\,.\,$ is bit-concatenation, $x$ is a given $28$-bit number, $k$ is a $36$-bit
number you can choose freely and $\texttt{hash}$ is the function defined below.
A different way of writing the same thing would be $\texttt{hash}(2^{36} x +
k)$.

The industrial management students were confident they had the resources and the
expertise to create a hash function better than SHA256. One of their genius
ideas was to use a $64$-bit input and output to save storage space. The result
of their extensive research, the hash function used in CC, is defined by the
following python code.

\begin{verbatim}
def hash(x):
    # int to list of bits
    def tolist(x):
        return [(x>>i)&1 for i in range(63, -1, -1)]

    # list of bits to int
    def toint(x):
        return sum([x[63-i]<<i for i in range(0, 64)])

    assert 0 <= x < 2**64
    x = tolist(x)
    for i in range(512):
        y = 4*x[i%64] + 2*x[(i+2)%64] + x[(i+10)%64]
        x[i%64] = [1,1,0,0,1,1,0,1][y]
    x = toint(x)
    for i in range(16):
        x += toint(tolist(x)[::-1])
        x ^= 3**40
        if(x >= 2**64): x >>= 1
    assert 0 <= x < 2**64
    return x
\end{verbatim}

\noindent You want to get rich by mining this coin. The Chalmers Coin
automatically changes its difficulty (how small the hash needs to be).
Therefore, you want to write a program which finds a value of $k$ to make the
hash as small as possible, the smaller the better.

\section*{Input}
Input consists of $10$ lines, each describing a block you want to mine. The
$i$th line contains the integer $x_i$. All $x_i$ are sampled uniformly at random
from $[0, 2^{28}-1]$.

\section*{Output}
Output $10$ integers $k_i$, one per line.

\section*{Scoring}

Your program will only be run on one test case (containing $10$ lines). The
sample test case is not worth any points. Your final score is the sum of your
score for each block, rounded to two decimal places. For each block, your score
is $\max(2t_i, 30t_i-20)$ where $t_i = \frac{64-\log_2(h_i)}{64-\log_2(H_i)}$,
$h_i$ is the hash you have obtained ($h_i = \texttt{hash}(2^{36} x_i + k_i)$)
and $H_i$ is the optimal hash.

TL;DR: You will get more points the smaller hashes you get.

\section*{Explanation of sample}
For the first block, the sample output gets the hash $h_1 = \texttt{hash}(2^{36}
\cdot 111131462 + 50363195222) = \texttt{hash}(7636895967909863254) =
15659998746091827$.  An optimal answer for the first block would have been
$1090594820$, corresponding to the hash $H_1 = 602833507760528$.  This gives
$t_1 = \frac{64-\log_2(h_1)}{64-\log_2(H_1)} \approx 0.68465$ and a score for
the first block of approximately $\max(2 \cdot 0.68465, 30 \cdot 0.68465-20) =
1.3693$.
